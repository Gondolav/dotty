In today\textquotesingle s world, we witness an increasingly need for type safety. Software is more complex than ever, and programmers want to be able to create interfaces with enhanced static information \cite{why-dependent-types-matter}. The goal is to increase the reliability and security of programs by specifying and enforcing rich data invariants \cite{dynamic-typing-with-dependent}. In other words, we seek domain-specific type checkers that could help us catching more errors at compile time, instead of waiting for them to surface at run-time.

With this goal in mind, the expressiveness of the type system can be increased by enabling type-level programming. In recent years, there has been considerable interest in the research community for this topic, and two solutions have mainly stood up: dependent types and type families.

Dependent types are types whose definition depends on a value. They can be used to maintain a data structure\textquotesingle s type as precise as its implementation. They have been widely adopted in proof assistants such as Agda \cite{agda}, Coq \cite{coq}, Epigram \cite{epigram} and Idris \cite{idris}. Although almost absent from general-purpose programming languages, they are starting to be partially adopted in Haskell \cite{dependent-types-eisenberg}, JavaScript \cite{dependent-javascript} and Scala\textquotesingle s next generation compiler, Dotty \cite{dependent-types-for-humans}. 

Type families are parametric types that can be assigned specialized representations based on the type parameters they are instantiated with. Being functions on types, they can be used, for example, to dynamically transform a function\textquotesingle s return type based on its arguments\textquotesingle \space types. Match types \cite{match-types-website}, under active development, are, with few differences, Dotty\textquotesingle s version of Haskell closed type families \cite{type-families}; they share some similarities with TypeScript\textquotesingle s conditional types too \cite{typescript-conditional}.

In this  work, we analyze dependent and match types, Dotty\textquotesingle s new features, by asserting their usability, both in terms of programming experience and performance, in the context of the development of a strongly-typed library for regular expressions. Inspired by Weirich\textquotesingle s talk \cite{wierich-talk}, our library has, at compile time, full knowledge of the types of the capturing groups inside a regex. For example, the simple regex "[0-9]", once compiled, has type \textit{String => Option[Int]}. This allows the type system to provide stronger guarantees to the programmer, which can specify the type of the desired regex and exploit the additional safety, given by the type checker, at compile time.

This is the first work, to our  knowledge, to use Dotty\textquotesingle s match and dependent types, still under development, in a relatively complex program. It is our hope that the present report will help Dotty developers\textquotesingle \space task and serve as a reference for future users.

The rest of the report is organised as follows. In Section 2, we illustrate with a concrete example from the library how these new technologies work. In Section 3, we describe in details the library implementation, showing how these features can be used. In Section 4, we present two possible use cases that demonstrate how dependent and match types can improve type safety. In Section 5, we compare the two implementations, highlighting strengths and weaknesses. In Section 6, we conclude with a description of possible extensions to our  work.

The source code for the regex library is available online\footnote{\url{https://github.com/Gondolav/dotty/tree/match-types-regex/strongly-typed-regex}}.
